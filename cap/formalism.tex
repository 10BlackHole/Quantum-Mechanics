\chapter{Formalism}
\section{Hilbert Space}

The purpose of this chapter is to recast the theory in a more powerful form.

Quantum theory is based on two constructs: \textit{wave functions} and \textit{operators}. The state of a system is represented by its wave function, obervables are represented by operators. Mathematically, weave functions satify the defining conditions for abstract \textbf{vectors}, and operators act on them as \textbf{linear transformations}. So the natural language of quantum mechanics is \textbf{linear algebra}.

But it is not, I suspect, a form of linear algebra with which you are immediately familiar. In an $N$-dimensional space it is simplest to represent a vector $\ket{\alpha}$, by the $N$-tuple of its components, $\{a_n\}$, with respect to a specified orthonormal basis:
\begin{equation}\label{3.1}
	\ket{\alpha}\to\vb*{a}=\mqty(a_1\\a_2\\\vdots \\a_N)
\end{equation}
The \textbf{inner product}, $\braket{\alpha}{\beta}$, of two vectors (generalizng the dot product in three dimensions) is the complex numebr,
\begin{equation}\label{3.2}
	\braket{\alpha}{\beta}=a_1^*b_1+a_2^*b_2+\cdots +a_N^*b_N
\end{equation}
Linear transformations, $T$, are represented by \textbf{matrices} (with respect to the specified basis), which act on vectors (to produce new vectors) by the ordinary rules of matrix multiplication:
\begin{equation}\label{3.3}
	\ket{\beta}=T\ket{\alpha}\to \vb{b}=\vb{T}\vb{a}=\mqty(t_{11}&t_{12}&\cdots & t_{1N}\\
	t_{21}&t_{22}&\cdots&t_{2N}\\
	\vdots&\vdots&\vdots&\vdots\\
	t_{N1}&t_{N2}& \cdots &t_{NN})
\end{equation}
But the "vectors" we encounter in quantum mechanics are (for the most specified) \textit{functions}, and they live in infinite-dimensional spaces. For them the $N$-tuple/matrix notation is awkward, at best, and manipulations that are well-behaved in the finite dimensional case can be problematic. (The underlying reason is that whereas the finite sum in (\ref{3.2}) always exists, an infinite sum--or a integral-- may not converge, in which case the inner product does not exists, and any argument involving inner products is immediately suspect.).

The collection of all functions of $x$ constitutes a vector space, but fot our purposes it is much too latge. To represent a possible physical state, the function $\Psi$ must be normalized:
$$\int |\Psi|^2\dd x=1$$ The set of all \textbf{square-integrable functions}, on a specified ineterval\footnote{Fot is, the limits ($a$ and $b$) will almost always be $\\pm\infty$, but we might as well keep thing more general for the moment},
\begin{equation}\label{3.4}
	f(x)\qquad \mbox{ such that  }\qquad \int_a^b |f(x)|^2\dd x<\infty
\end{equation}
constitutes a (much smaller) vector space. Mathematicians call it $L_2(a,b)$; physicist call it \textbf{Hilbert space}. In quantum mechanics then
\begin{equation}\label{3.5}
	\boxed{\mbox{\textbf{Wave functions live in Hilbert space.}}}
\end{equation}
We define the \textbf{inner product of two functions}, $f(x)$ and $g(x)$, as follows:
\begin{equation}\label{3.6}
	\braket{f}{g}\equiv\int_a^b f(x)^*g(x)\dd x
\end{equation}
If $f$ and $g$ are both square-integrable (that is, if they are both in hilber spaces), their inner product is guaranteed to exist. This follows from the integral \textbf{Schwartz inequality}:
\begin{equation}\label{3.7}
	\left|\int_a^bf(x)^*g(x)\dd x\right|\leq \sqrt{\int_a^b|f(x)|^2\dd x\int_a^b|g(x)|^2\dd x}
\end{equation}
Notice in particular that
\begin{equation}\label{3.8}
	\braket{g}{f}=\braket{f}{g}^*
\end{equation}
Moreover, the inner product of $f(x)$ with itself,
\begin{equation}\label{3.9}
	\braket{f}{f}=\int_a^b|f(x)|^3\dd x
\end{equation}
is real and and non-negative; it's zero only when $f(x)=0$. 

A function is said to be \textbf{normalized} if its inner product with itself is $1$; two functions are \textbf{orthogonal} of their inner product is $0$; and a set of functions, $\{f_n\}$, is \textbf{orthonormal} of they are normalized and mutually orthogonal
\begin{equation}\label{3.10}
	\braket{f_m}{f_n}=\delta_{mn}
\end{equation}
Finally, a set of functions is \textbf{compelte} if any other function (in Hilbert space) can be expreded as a linear combination of them:
\begin{equation}\label{3.11}
	f(x)=\sum_{n=1}^\infty c_nf_n(x)
\end{equation}
If rge functuons $\{f_n(x)\}$ are orthonrmal, the coefficients are given by Fourier's trick
\begin{equation}\label{3.12}
	c_n=\braket{f_n}{f}
\end{equation}

\section{Obervables}
\subsection{Hermitian Operators}
The expectation value of an observable $Q(x,p)$ can be expressed very neatly in inner-product notation:
\begin{equation}\label{3.13}
	<Q>=\int \Psi^*\hat{Q}\Psi\dd x=\bra{\Psi}\ket{\hat{Q}\Psi}
\end{equation}
Now, the outcome of a measurement has got to be \textit{real}, and so, a fortiori, is the average of many measurements:
\begin{equation}\label{3.14}
	<Q>=<Q>^*
\end{equation}
But the complex conjugate of an inner product reverses the order (\ref{3.8}), so
\begin{equation}\label{3.15}
	\bra{\Psi}\ket{\hat{Q}\Psi}=\bra{\hat{Q}\Psi}{\ket\Psi}
\end{equation}
and this must hold true for any wave function $\Psi$. Thus operators representing \textit{obervables} have the very special property that
\begin{equation}\label{3.16}
	\bra{f}\ket{\hat{Q}f}=\bra{\hat{Q}f}\ket{f}\qquad\mbox{for all $f(x)$}
\end{equation}
We call such operators \textbf{hermitian}.

The essentiak point is that a hermitian operator can be applied either to the first member of an inner product or to the second, with the same result, and hermitian operators naturally arise in quantum mechanics because their expectations values are real:
\begin{equation}\label{3.18}
	\boxed{\mbox{\textbf{Obervables are represented by hermitian operators.}}}
\end{equation}
Is the momentum operator, for example, hermitian?
\begin{equation}\label{3.19}
	\bra{f}\ket{\hat{p}g}=\int_{-\infty}^\infty \frac{\hbar}{i}\dv{g}{x}\dd x=\frac{\hbar}{i}\eval{f^*g}_{-\infty}^\infty +\int_{-\infty}^\infty \left(\frac{\hbar}{i}\dv{f}{x}\right)^*g\dd x=\bra{\hat{p}f}\ket{g}
\end{equation}
Notice if $f(x)$ and $g(x)$ are squared integrable, they must go to zero at $\pm\infty$.

\subsection{Determinate States}
Ordinarily, when you measure an obervable $Q$ on an ensamble of identically prepared system, all in the same state $\Psi$, you do not ger the same reuslt each time--this is the indeterminancy of quantum mechanics. \textit{Question:} Would it be possible to prepare a state such that \textit{every} measurement of $Q$ is certain to return the \textit{same} value (call it $a$)? This would be, if younlike, a \textbf{determinate state}, for the obervable $Q$ (Actually, we already know one example: Stationary states are dterminate states of Hamiltonian; a measurement of the total enery, on a particle in the stationary state $\Psi_n$, is certain to yield the corresponding "allowed" energy $E_n$).

Well, the standard deviation of $Q$, in a determiante state, would to be zero, shich is to say,
\begin{equation}\label{3.21}
	\sigma^2=<(\hat{Q}-<Q>)^2>=\bra{\Psi}\ket{(\hat{Q}-q)^2\Psi}=\bra{(\hat{Q}-q)\Psi}\ket{(\hat{Q}-q)\Psi}=0
\end{equation}
But the only function whose inner product with itself vanishes is $0$, so
\begin{equation}\label{3.22}
	\hat{Q}\Psi=q\Psi
\end{equation}
This is the \textbf{eigenvalue equation} for the operator $\hat{Q}$; $\Psi$ is an \textbf{eigenfunction} of $\hat{Q}$, and $q$ is the corresponding \textbf{eigenvalue}. Thus
\begin{equation}\label{3.23}
	\boxed{\mbox{\textbf{Determiante states are eigenfunctions of }$\hat{Q}$}}
\end{equation}
Measurements of $Q$ a¡on such states is certain to yield the eigenvalue, $q$.

Note that the eigenvalue is a number (not a operator or a function). You can multuply any eigenfunction by a constant,  and it is still an eigenfunction, with the same eigenvalue. Zero does not count as an eigenfunction. But there's nothing wrong with zero as an eigenvalue. The collection of all the eigenvalues of an operator is called its \textbf{specturm} . Sometimes two (or more) linearly independent eigenfunctions share the same eigenvalue; in that case the spectrum is said to be \textbf{degenerate}.

\section{Eigenfunctions of a Hermitian Operator}
Our attention is thus directed to the \textit{eigenfunction of hermitian operators} (physically: determine states of obervables). These fall into two categories: If the spectrum is \textbf{discrete} (i.e., the eigenvalues are separated from one another) then the eigenfunctions lie in Hilbert space and they constitute physically realizable states. If the spectrum is \textbf{contonuous} (i.e., the eigenvalues fill out an entire range) then the eigenfunctions are non normalizable, and they do not represent possible wave functions (\textit{though linear cominations} of them--involving neccessarily a spread in eigenvalues--may be normalizable). Some operator have a discrete spectrum only (for example, the Hamiltonian for the harmonic oscillator). some have only a continuous spectrum (for example, the free partivle Hamiltonian), and some have both discrete part and a continuous part (for example, the Hamiltonian for a finite square well). The discrete case is easier to handle. because the relevant inner product are guaranteed to exist--in fact, it is very similar to the finite-dimensional theory (the eigenvectors of a hermitian matrix). I'll treat the discrete case first, and then the continuous one:

\subsection{Discrete Spectra}
Mathematically, the normalizable eigenfunctions of a hermitian have two important properties:
\begin{teo}
	Their \textit{eqigenvalues} are \textit{real}.
\end{teo}
This is comforting: If you measure an obervable on a particle ina  dterminate state, you will at leat least get a real number.
\begin{teo}
	Eigenfunctions belonging to distinct eigenvalues are \textit{orthogonal}.
\end{teo}
In  a \textit{finite}-diemsnional vector space the eigenvectos of a hermitian matrix have a third fundamental property: The span the space (every vector can be expressed as a linar combination of them). Unfortunately, the proof does not generalize to infinite-dimensional spaces. But the property itslef is essential to the internal consistency of quantum mechanics, so we will take it as an \textit{axiom} (or, more precisely, as a restriction on the class of hermitian operators that can represent obervables):
\begin{axiom}
	The eigenvalues of an obervable operator are \textit{compelte}: Any function (in Hilber space) can be expressed as a linear combination of them.
\end{axiom}

\subsection{Continuous Spectra}
If the spectrum of a hermitian operator is \textit{continuous}, the eigenfunctions are not normalizable, and the proofs of Theorems 1 and 2 fail, because the inner products may not exists. Nevertheless, there is a sense in which the three essential properties (reality, orthogonality, and complereness) still hold. I think it's best to approach this subtle case through specific examples.

\begin{example}
	Find the eigenfunctions and eigenvalues of the momentum operator
\end{example}
\begin{sol}
\end{sol}
Let $f_p(x)$ be the eigenfunction and $p$ the eigenvalue:
\begin{equation}\label{3.30}
	\frac{\hbar}{i}\frac{\dd}{\dd x}f_p(x)=pf_p(x)
\end{equation}
The general solution is $$f_p(x)=Ae^{ipc/\hbar}$$ This is not square-integrable, for any (complex) value of $p$--the momentum operator has no eigenfunctions in Hilbert space. And yet, if we restrict ourselves to \textit{real} eigenvalues. we do recover a kind of \textit{ersatz} "orthonormality". ...

\subsection{Generalized Statistical Interpretation}














