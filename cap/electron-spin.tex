\chapter{Electron Spin}\footnote{Quantum Mechanics - Cohen-Tannoudji, Diau, Laloe.}
\section{Introduction of electron spin}
\subsection{Experimental evidence}
\subsubsection{Fine structure of spectral lines}
\subsubsection{"Anomalous" Zeeman effect}
\subsubsection{Existence of half-integral angular momenta}

\subsection{Quantum description: postulates of the Pauli theory}

\section{Special properties of an angular momentum $1/2$}
We shall restrict ourselves from now on to the case of the electron, which is a spin $1/2$ particle. From the preceding chapters, we know how to handle its orbital variables. We are now going to study in more detail its spin degrees of freedom.

The spin state space is two-dimensional. We shall take as a basis the orthonormal system $\{\ket{+}, \ket{-}\}$ of eigenkets common to $\vb{S}^2$and $S_z$ which satisfy the equations:

\begin{equation}\label{B-1}
	\left \{
		\begin{array}{ll}
			\vb{S}^{2}\ket{\pm}&=\frac{3}{4}\hbar^2\ket{\pm}\\
			S_z\ket{\pm}&=\pm\frac{1}{2}\hbar\ket{\pm}
		\end{array}
	\right.
\end{equation}

\begin{equation}\label{B-2}
	\left \{
		\begin{array}{ll}
			\braket{+}{-}&=0\\
			\braket{+}{+}&=\braket{-}{-}=1
		\end{array}
	\right.
\end{equation}

\begin{equation}\label{B-3}
	\ket{+}\bra{+} + \ket{-}\bra{-}=\mathbb{1}
\end{equation}
where $\mathbb{1}$ is the unit operator. The most general spin state is described by an arbitrary vector of $\mathcal{E}_c$:
\begin{equation}\label{B-4}
	\ket{\xi}=c_+\ket{+} + c_-\ket{-}
\end{equation}
where $c_+$ and $c_-$ are complex numbers. According to (\ref{B-1}), all the kets of $\mathcal{E}_s$ are eigenvectors of $\vb{S}^2$with the same eigenvalue $3\hbar^2/4$, which causes $\vb{S}^2$ to be proportional to the identity operator of $\mathcal{E}_s$:
\begin{equation}\label{B-5}
	\vb{S}^2=\frac{3}{4}\hbar^2
\end{equation}
(in the right hand side of this equation, as is usually done, we have not written the unit operator $\mathbb{1}$ explicitly). Since $\vb{S}$ is, by definition, an angular momentum, it possesses all the general properties derived before. The action of the operators:
\begin{equation}\label{B-6}
	S_{\pm}=S_x\pm iS_i
\end{equation}
on the basis vectors $\ket{+}$ and $\ket{-}$ is given by the general formulas (\ref{C-50}) when one sets $j=s=1/2$:
\begin{align}
	\label{B-7a}S_-\ket{+}&=0\qquad S_+\ket{-}=\hbar\ket{+}\\
	S_-\ket{+}&=\hbar\ket{-}\qquad S_-\ket{-}=0\label{B-7b}
\end{align}
Any operator acting in $\mathcal{E}_s$ can be representes, in the $\{\ket{+}, \ket{-}\}$ basis, by a $2\times 2$ matrix. In particular, using (\ref{B-1b}) and (\ref{B-7}), we find the matrices corresponding to $S_x, S_y$ and $S_z$ in the form:
\begin{equation}\label{B-8}
	(\vb{S})=\frac{\hbar}{2}\vb*{\sigma}
\end{equation}
where $\vb{\sigma}$ designates the set of three \textit{Pauli matrices}:
\begin{equation}\label{B-9}
	\sigma_x=\mqty(0&1\\1&0)\qquad \sigma_y=\mqty(0&-i\\i&0)\qquad  \sigma_z=\mqty(1&0\\0&-1)
\end{equation}
The Pauli matrices posses the following properties, which can easily be verifed from their explicit form (\ref{B-9})
\begin{align}
	\label{B-10a}\sigma_x^2=\sigma_y^2=\sigma_z^2&=\mathbb{1}\\
	\label{B-10b}\sigma_x\sigma_y + \sigma_y\sigma_x&=0\\
	\label{B-10c}[\sigma_x,\sigma]&=2i\sigma_z\\
	\sigma_x\sigma_y&=i\sigma_z\label{B-10d}
\end{align}
(to the last three formulas must bee added those obtained through cyclic permutation of the $x,y,z$ indices). It also follow from (\ref{B-9}) that:






