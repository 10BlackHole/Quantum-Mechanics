\chapter{Quantum Mechanics In Three Dimensions}
\section{Schrodinger Equation In Spherical Coordinates}
The generalization to three dimensions is straightforward. Schrodinger's equation says
\begin{equation}\label{4.1}
	i\hbar\pdv{\Psi}{t}=H\Psi
\end{equation}
the Hamiltonian operator\footnote{Where confusion might otherwise occur I have been putting "hats" on operators, to distinguish the from the corresponding classical observables. I don't think therw will be much occasion for ambiguity in this chapter, and the hats get to be cumbersome, so I going to leave the off from now on.} $H$ is obtained from the classical energy $$\frac{1}{2}mv^2+V=\frac{1}{2}(p_x^2+p_y^2+p_z^2)+V$$ by the standard prescription (applied now to $y$ and $z$, as well as $x$):
\begin{equation}\label{4.2}
	p_x\to\frac{\hbar}{i}\frac{\partial}{\partial x},\quad p_y\to\frac{\hbar}{i}\frac{\partial}{\partial y},\quad p_z\to\frac{\hbar}{i}\frac{\partial}{\partial z}
\end{equation}
or
\begin{equation}\label{4.3}
	\boxed{\vb{p}\to\frac{\hbar}{i}\nabla}
\end{equation}
for short. Thus
\begin{equation}\label{4.4}
	\boxed{i\hbar\pdv{\Psi}{t}=-\frac{\hbar^2}{2m}\nabla^2\Psi + V\Psi}
\end{equation}
where
\begin{equation}\label{4.5}
	\nabla^2\equiv \frac{\partial^2}{\psi x^2} + \frac{\partial^2}{\partial y^2} + \frac{\partial^2}{\partial z^2}
\end{equation}
is the \textbf{Laplacian}, in cartesian coordinates.

The potential energy $V$ and the wave function $\Psi$ are now functions of $\vb{r}=(x,y,z)$ and $t$. The probability of finsing the particle in the infinitesimal volume $\dd ^3\vb{r}=\dd x\dd y\dd z$ is $|\Psi(\vb{r},t)|^2\dd^3\vb{r}$,  and the normalization condition reads
\begin{equation}\label{4.6}
	\int |\Psi|^2\dd ^3\vb{r}=1
\end{equation}
with the integral taken over all space. If the potential is independent of time, there will be a complete seo of stationary states
\begin{equation}\label{4.7}
	\Psi_n(\vb{r},t)=\psi_n(\vb{r})e^{-iE_nt/\hbar}
\end{equation}
where the spatial wave function $\psi_n$ stisfies the time-\textit{independent} Schrodinger equation:
\begin{equation}\label{4.8}
	\boxed{-\frac{\hbar^2}{2m}\nabla^2\psi+V\psi=E\psi}
\end{equation}
The general solution to the (time-\textit{dependent}) Schrodinger equation is
\begin{equation}\label{4.9}
	\Psi(\vb{r},t)=\sum c_n\psi_n(\vb{r})e^{-iE_nt/\hbar}
\end{equation}
where de constants $c_n$ determined by the initial wave function


 


 

 

 





























\section{The Hydrogen Atom}
The hydrogen atom consists of a heavy, essentially motionless proton (we may as well put it at the origin), of charge $e$, together with a much lighter electron (charge $-e$) that orbits around it, bound by the mutual attraction of opposite charges. From Coulomb's law, the potential energy (in SI units) is
\begin{equation}\label{4.52}
	V(r)=-\frac{e^2}{4\pi\epsilon_0}\frac{1}{r}
\end{equation}
and the radial equation (\ref{4.37}) says
\begin{equation}\label{4.53}
	-\frac{\hbar^2}{2m}\dv[2]{u}{r}+\left[-\frac{e^2}{4\pi\epsilon_0}\frac{1}{r}+\frac{\hbar^2}{2m}\frac{l(l+1)}{r^2}\right]u=Eu
\end{equation}
Our problem is to solve this equation for $u(r)$, and determine the allowed energies, $E$. The hydrogen atom is such an important case that I'm not going hand you the solutions this time--we'll work them out in detail, by the method we used in the analytica solution to the harmonic oscillator. 

Insidentally, the Coulomb potential (\ref{4.53}) admits \textit{continuum} states (with $E>0$), describing electron-proton scattering, as well as discrete \textit{bound} states, representing the hydrogen atom, but we shall confine our attention to the latter.

\subsubsection{The Radial Wave Function}
Our first task is to tidy up the notation. Let
\begin{equation}\label{4.54}
	\kappa\equiv\frac{\sqrt{-2mE}}{\hbar}
\end{equation}
(For bound states, $E$ is negative, so $\kappa$ is \textit{real}.) Dividing (\ref{4.53}) by $E$, we have $$\frac{1}{\kappa^2}\dv[2]{u}{r}=\left[1-\frac{me^2}{2\pi\epsilon_0\hbar^2\kappa}\frac{1}{(\kappa r)}+\frac{l(l+1)}{(\kappa r)^2}\right]u$$ This suggests that we introduce
\begin{equation}\label{4.55}
	\rho\equiv\kappa r,\quad \mbox{and}\quad \rho_0\equiv\frac{me^2}{2\pi\epsilon_0\hbar^2\kappa}
\end{equation}
so that
\begin{equation}\label{4.56}
	\dv[2]{u}{\rho}=\left[1-\frac{\rho_0}{\rho}+\frac{l(l+1)}{\rho^2}\right]u
\end{equation}
Next we examine the asymptotic form of the solutions. As $\rho\to \infty$, the constant term in the brackets dominates, so (approximately) $$\dv[2]{u}{\rho}=u$$ The general solution is
\begin{equation}\label{4.57}
	u(\rho)=Ae^{-\rho}+Be^{\rho}
\end{equation}
but $e^{\rho}$ blows up (as $\rho\to\infty$), so $B=0$. Evidently
\begin{equation}\label{4.58}
	u(\rho)\thicksim Ae^{-\rho}
\end{equation}
for large $\rho$. On the other hand, as $\rho\to 0$ the centrifugal term dominates\footnote{This argument does not apply when $l=0$ (although the conclusion, (\ref{4.59}, is in fact valid for that case too). But never mind: All I am trying to do is provide some \textit{motivation} for a change of variables (\ref{4.60}))}; approximately, then: $$\dv[2]{u}{\rho}=\frac{l(l+1)}{\rho^2}u$$ The general solution (check it!) is $$u(\rho)=C\rho^{l+1}+D\rho^{-l}$$ but $\rho^{-1}$ blows op (as $\rho\to 0$), so $D=0$. Thus
\begin{equation}\label{4.59}
	u(\rho)\thicksim \rho^{l+1}
\end{equation}
for small $\rho$.

The next step is to peel off the asymptotic behavior, introducing the new function $v(\rho)$:
\begin{equation}\label{4.60}
	u(\rho)=\rho^{l+1}e^{-\rho}v(\rho)
\end{equation}
in the hope that $v(\rho)$ will turn out to be simpler than $u(\rho)$. The first indications are not auspicious: $$\dv{u}{\rho}=\rho^le^{-\rho}\left[(l+1-\rho)v+\rho\dv{v}{\rho}\right]$$ and $$\dv[2]{u}{\rho}=\rho^le^{-\rho}\left\{\left[-2l-2+\rho+\frac{l(l+1)}{\rho}\right]v+2(l+1-\rho)\dv{v}{\rho}+\rho\dv[2]{v}{\rho}\right\}$$ In terms of $v(\rho)$, then, the radial equation (\ref{4.56}) reads
\begin{equation}\label{4.61}
	\rho\pdv[2]{v}{\rho}+2(l+1-\rho)\dv{v}{\rho}+[\rho_0-2(l+1)]v=0
\end{equation}
Finally, we assume the solution, $v(\rho)$, can be expressed as a power series in $\rho$:
\begin{equation}\label{4.62}
	v(\rho)=\sum_{j=1}^\infty c_j\rho^j
\end{equation}
Our problem is to determine the coefficients ($c_0, c_1, c_2,...$). Differentiating term by term $$\dv{v}{\rho}=\sum_{j=0}^\infty jc_j\rho^{j-1}=\sum_{j=0}^\infty(j+1)c_{j+1}\rho^j$$
[In the second summation I have renamed the "dummy index": $j\to j+1$. If this troubles you, write out the first few terms explicitly, and \textit{check} it. You may object that the sum should now begin at $j=-1$, but the factor $(j+1)$ kills that term anyway, so we might as well start at zero.] Differentiating again $$\dv[2]{v}{\rho}=\sum_{j=1}^\infty j(j+1)c_{j+1}\rho^{j-1}$$ Inserting these into (\ref{4.61}), we have $$\sum_{j=1}^\infty j(j+1)c_{j+1}\rho^j+2(l+1)\sum_{j=0}^\infty (j+1)c_{j+1}\rho^j$$ $$-2\sum_{j=0}^\infty jc_j\rho^j+[\rho_0-2(l+1)]\sum_{j=0}^\infty c_j\rho^j=0$$ Equating the coefficients of like powers yields $$j(j+1)c_{j+1}+2(l+1)(j+1)c_{j+1}-2cc_j+[\rho_0-2(l+1)]c_j=0$$ or 
\begin{equation}\label{4.63}
	c_{j+1}=\left\{\frac{2(j+l+1)-\rho_0}{(j+1)(j+2l+2)}\right\}c_j
\end{equation}
This recursion formular determines the coefficients, and hence de function $v(\rho)$: We start with $c_0$ (this becomes an overall constant, to be fixde eventually by normalization), and (\ref{4.63}) gives is $c_1$; putting this back in, we obtain $c_2$, and so on\footnote{You might wonder why I didn't use the series method directly on $u(\rho)$--why factor out the asymptotic behavior before applying this procedure? Well, the reason for peeling off $\rho^{l+1}$ is largely aesthetic: Without this, the sequnce would begin with a long string of zeris (the first nonzero coefficient begin $c_{l+1}$): by factoring out $\rho^{l+1}$ we obtain a series that starts out with $\rho^0$. The $e^{-\rho}$ factor is more critical--if you don't pull that out, you get a three-term recursion formula, involving $c_{j+1}, c_{j+1}$, and $c_j$  and that is enormously more difficult to work with.}.

Now let's see what the coefficients look like for large $j$ (this corresponds to large $\rho$, where the higher powers dominate). In this regime the recursion formula says $$c_{j+1}\approx \frac{2j}{j(j+1)}c_j=\frac{2}{j+1}c_j$$ Suppose for a moment that this were \textit{exact}. Then 
\begin{equation}\label{4.64}
	c_j=\frac{2^j}{j!}c_0
\end{equation}
so $$ v(\rho)=C_0\sum_{j=0}^\infty \frac{2^j}{j!}\rho^j=c_0e^{2\rho}$$ and hence 
\begin{equation}\label{4.65}
	u(\rho)=c_0\rho^{l+1}e^\rho
\end{equation}
which blows uo at large $\rho$. The positive exponential is precisely the asymptotic behavior we didn't want, in (\ref{4.57}). (It's no accident that it reappears here; after all, it does represent the asymptotic form of \textit{some} solutions to the radial equation--they just don't happen to be the ones we're interested in, because thay aren't normalizable.) There is only one way out of this dilema: The \textit{series must terminate}. There must occur some maximal integer, $j_{max}$, such that
\begin{equation}\label{4.66}
	c(j_{max}+1)=0
\end{equation}
(and beyond which all coefficients vanish automatically). Evidently (\ref{4.63}) $$2(j_{max}+l+1)-\rho_0=0$$ Defining
\begin{equation}\label{4.67}
	n\equiv j_{max} +l+1
\end{equation}





























\section{Angular Momentum}

Classically, the angular momentum of a particle (with respecto to the origin) is given by the formula
\begin{equation}\label{4.95}
	\vb{L}=\vb{r}\times\vb{p}
\end{equation}
which os to say,
\begin{equation}\label{4.96}
	L_x=yp_z-zp_y,\quad L_y=zp_x-xp_z,\quad L_z=xp_y-yp_x
\end{equation}
The corresponding quantum operators are obtained by the standard prescription $p_x\to -\hbar\partial/\partial x, p_y\to -i\hbar\partial/\partial y, p_z\to -\hbar\partial/\partial z$.

\subsubsection{Eigenvalues}
The operators $L_x$ and $L_y$ do not commute; in fact\footnote{Note that all the operators we encounter in quantum mechanics are distributive with respect to addition: $A(B+C)=AB +AC$. In particular, $[A,B+C]=[A,B]+[A,C]$.}
\begin{align}
	\nonumber [L_x,L_y]&=[yp_z-zp_y, zp_x-xp_z]\\
										 &=[yp_z,zp_x]-[yp_z,zp_z]-[zp_y,zp_x]+[zp_y,xp_z]\label{4.97}
\end{align}
From the canonical commutation relations\footnote{The cannonical commutation relations are $[r_i,p_j]=-[p_i,r_j]=i\hbar\delta_{ij}\quad [r_i,r_j]=[p_i,p_j]=0$} we know that the inly operators here that \textit{fail} to commute are $x$ with $p_x$, $y$ with $p_y$, and $z$ with $p_z$. So the two middles terms drop out, leaving
\begin{equation}\label{4.98}
	[L_x,L_y]=yp_x[p_z,z]+xp_y[z,p_z]=i\hbar(xp_y-yp_x)=i\hbar L_z
\end{equation}
Of course, we could have started out with $[L_y,L_z]$ or $[L_z,L_x]$, but there is no need to calculate these separately--we can get them immediately by cyclic permutation of the indices ($x\to y, y\to z, z\to x$)
\begin{equation}\label{4.99}\marginnote{Fundamental commutation relations for angular momentum}
	\boxed{[L_x,L_y]=i\hbar L_z;\quad [L_y,L_z]=i\hbar L_x;\quad [L_z,L_x]=i\hbar L_y}
\end{equation}
These are the fundamental commutation relations for angular momentum; everything else follows from them.

Notice that $L_x, L_y$ and $L_z$ are \textit{incompatible} obervables. According to the generalized uncertainty principle (\ref{3.62})
\begin{equation*}
	\sigma_{L_x}^2\sigma_{L_y}^2\geq\left(\frac{1}{2i}\expval{i\hbar L_z}\right)^2=\frac{\hbar^2}{4}\expval{L_z}^2
\end{equation*}
or
\begin{equation}\label{4.100}
	\sigma_{L_x}\sigma_{L_y}\geq\frac{\hbar}{2}|\expval{L_z}|
\end{equation}
It would therefore be fuitle to look for states that are simultaneoulsy eigenfunctions of $L_x$ and $L_y$. On the other hand,  the square of the total angular momentum,
\begin{equation}\label{4.101}
	L^2\equiv L_x^2+L_y^2+L_z^2
\end{equation}
does commute with $L_x$:
\begin{align*}
	[L^2,L_x]&=[L_x^2,Lx]+[L_y^2,L_x]+[L_z^2,Lx]\\
					 &=L_y[L_y,L_x]+[L_y,L_x]L_y+L_z[L_z,L_x]+[L_z,L_x]L_z\\
					 &=L_y(-i\hbar L_z)+(-i\hbar L_z)L_y+L_z(i\hbar L_y)+(i\hbar L_y)L_z\\
					 &=
\end{align*}
(I used (\ref{3.64})) to simplify the commutator; note also that any operator commutes with itself. It follows, of course, that $L^4$ also commute with $L_y$ and $L_z$:
\begin{equation}\label{4.102}
	[L^2,L_x]=0,\quad [L^2,L_y]=0,\quad {L^2,L_z}=0
\end{equation}
or, more compactly, 
\begin{equation}\label{4.103}
	[L^2,\vb{L}]=0
\end{equation}
So $L^2$ is compatible with each component of $\vb{L}$, and we can hope to find simultanoeus eigenstates of $L^2$ and (say) $L_z$:
\begin{equation}\label{4.104}
	L^2f=\lambda f\quad \mbox{and}\quad L_zf=\mu f
\end{equation}
We'll use a "ladder operator" technique, very similar to the one we applied to the harmonic oscillator back in Section \ref{sec:2.3.1}. Let
\begin{equation}\label{4.105}
	\boxed{L_{\pm}\equiv L_x\pm iL_y}
\end{equation}
The commutator with $L_z$ is
\begin{equation*}
	[L_z,L_{\pm}]=[L_z,L_x]\pm i[L_z,L_y]=i\hbar L_y\pm i(-i\hbar L_x)=\pm\hbar (L_x\pm iL_y)
\end{equation*}
so
\begin{equation}\label{4.106}
	[L_z,L{\pm}]=\pm \hbar L_{\pm}
\end{equation}
And. of course, 
\begin{equation}\label{4.107}
	[L^2,L_{\pm}]=0
\end{equation}
I claim that if $f$ es an eigenfunction of $L^2$ and $L_z$, so also is $L_{\pm} f$: (\ref{4.107}) says
\begin{equation}\label{4.108}
	L^2(L_{\pm}f)=L_{\pm}(L^2 f)=L_{\pm}(\lambda f)=\lambda(L_{\pm}f)
\end{equation}
so $L_{\pm}f$ is an eigenfunction of $L^2$, with the same eigenvalue $\lambda$, and (\ref{4.106}) says
\begin{align}
\nonumber	L_z(L_{\pm}f)&=(L_zL_{\pm}-L_{\pm}L_z)f+L_{\pm}L_zf=\pm\hbar L_{\pm}f+L_{\pm}(\mu f)\\
											 &=(\mu\pm \hbar)(L_{\pm}f)\label{4.109}
\end{align}
so $L_{\pm} f$ is an eigenfunction of $L_z$ with the \textit{new} eigenvalue $\mu\pm\hbar$. We call $L_+$ the "raising" operator, because is \textit{increases} the eigenvalue of $L_z$ by $\hbar$, and $L_-$ the "lowering" operator, because it \textit{lowers} the eigenvalue by $\hbar$.

For a given value of $\lambda$, then we obtain a "ladder" of states, with each "rung" separated from its neighbors by one unit of $\hbar$ in the eigenvalue of $L_z$. To ascend the ladder we apply the raising operator, and to descend, the lowering operator. But this procces cannot go on forever. Eventually we're going to reach a state gor which the $z$-component exceeds the \textit{total}, and that cannot be. There must exist a "top rung", such that
\begin{equation}\label{4.110}
	L_+f_t=0
\end{equation}
Let $\hbar l$ be the eigenvalue of $L_z$ at this top ring (the appropriateness of the letter "$l$" will appear in a moment):
\begin{equation}\label{4.111}
	L_zf_t=\hbar lf_t;\quad L^2f_t=\lambda f_t
\end{equation}
Now,
\begin{align*}
	L_{\pm}L_{\mp}&=(L_x\pm iL_y)(L_x\mp iL_y)=L_x^2+L_y^2\mp i(L_xL_y-L_yL_x)\\
								&=L^2-L_z^2\mp i(i\hbar L_z)
\end{align*}
or, putting in the other way around,
\begin{equation}\label{4.112}
	L^2=L_{\pm}L_{\mp}+L_z^2\mp \hbar L_z
\end{equation}
It follows that
$$L^2f_t=(L_-L_++L_z^2+\hbar L_z)f_t=(0+\hbar^2l^2+\hbar^2l)f_t=\hbar^2l(l+1)f_t$$ and hence
\begin{equation}\label{4.113}
	\lambda = \hbar^2l(l+1)
\end{equation}
This tells is the eigenvalue of $L^2$ in terms of the \textit{maximum} eigenvalue of $L_z$.

Meanwhile, there is also (for the same reason) a \textit{bottom} rung, $f_b$, such that
\begin{equation}\label{4.114}
	L_-f_b=0
\end{equation}
Let $\hbar\bar{l}$ be the eugenvalue if $L_z$ at this bottom rung:
\begin{equation}\label{4.115}
	L_zf_b=\hbar\bar{l}f_b;\quad L^2f_b=\lambda f_b
\end{equation}
Using (\ref{4.112}), we have $$L^2f_b=(L_+L_-+L_z^2-\hbar L_z)f_b=(0+\hbar^2\bar{l}^2-\hbar^2\bar{l})f_b=\hbar^2\bar{l}(\bar{l}-1)f_b$$ and therefore
\begin{equation}\label{4.116}
	\lambda = \hbar^2\bar{l}(\bar{l}-1)
\end{equation}
Comparing (\ref{4.113}) and ({4.116}), we see that $l(l+1)=\bar{l}(\bar{l}-1)$, so either $\bar{l}=l+1$ (which is absurd--the bottom rung would be higher than the top rung!) or else
\begin{equation}\label{4.117}
	\bar{l}=-l
\end{equation}
Evidently the eigenvalues of $L_z$ are $m\hbar$ (the appropaites of this letter will aso be clear in a moment) goes from $-l$ to $+l$ in a $N$ integer steps. In a particular, it follows that $l=-l+N$, and hence $l=N/2$, so $l$ must be \textit{an integer or a half-integer}. The eigenfunctions are characterized by the numbers $l$ and $m$:
\begin{equation}\label{4.118}
	\boxed{L^2f_l^m=\hbar^2l(l+1)f_l^m;\quad L_zf_l^m=\hbar mf_l^m}
\end{equation}
where 
\begin{equation}\label{4.119}
	l=0,1/2,1,3/2,...;\quad m=-l, -l+1,..., l-1, l
\end{equation}
For a given value of $l$, there are $2l+1$ different values of $m$ (i.e., $2l+1$ "rungs" on the "ladder"). 

I hope you're impressed: By \textit{purely algebraic means}, startin with the fundamental commutation relations for angular momentum (\ref{4.99}), we have determined the eigenvalues of $L^2$ and $L_z$--without ever seeing the eigenfunctions themselves! We turn now to the problem of constucting the eigenfunctions. but we're headed. I'll begin with the punch line: $f_l^m=Y_l^m$ the eigenfunctions of $L^2$ and $L_z$ are nothing but the old spherical harmonics, whics we came upon by a quite different route (that's ehy I chose the letters $l$ and $m$, of course). And I can now tell you whi the spherical harmonics are orthogonal: They are eigenfucntions of hermitian operatos ($L^2$ and $L_z$) belonging to distinct eigenvalues.

\subsubsection{Eigenfunctions}\label{sec:4.3.2}
First of all we need no rewrite $L_x, L_y$ and $L_z$ in spherical cooridinates. Now, $\vb{L}=(\hbar/i)(\vb{r}\times \nabla)$, and the gradient, in spherical cooridinates, is:
\begin{equation}\label{4.123}
	\nabla=\hat{r}\frac{\partial}{\partial r}+\hat{\theta}\frac{1}{r}\frac{\partial}{\partial \theta}+\hat{\phi}\frac{1}{r\sin\theta}\frac{\partial}{\partial \phi}
\end{equation}
meanwhile, $\vb{r}=r\hat{r}$, so $$\vb{L}=\frac{\hbar}{i}\left[r(\hat{r}\times\hat{r})\frac{\partial}{\partial r}+(\hat{r}\times\hat{\theta})\frac{\partial}{\partial\theta} + (\hat{r}\times\hat{\phi})\frac{1}{\sin\theta}\frac{\partial}{\partial\phi}\right]$$. But $(\hat{r}\times\hat{r})=0$, $(\hat{r}\times\hat{\theta})=\hat{\phi}$ , and $(\hat{r}\times\hat{\phi})=-\hat{\theta}$, and hence
\begin{equation}\label{4.124}
	\vb{L}=\frac{\hbar}{i}\left(\hat{\phi}\frac{\partial}{\partial\theta}-\hat{\theta}\frac{1}{\sin\theta}\frac{\partial}{\partial\phi}\right)
\end{equation}
The unit vectors $\hat{\theta}$ and $\hat{\phi}$ can be resolved into their cartesian components:
\begin{equation}\label{4.125}
	\hat{\theta}=(\cos\theta\cos\phi)\hat{i } + (\cos\theta\sin\phi)\hat{j} + (\sin\theta)\hat{k}
\end{equation}
\begin{equation}\label{4.126}
	\hat{\phi}=-(\sin\theta)\hat{i} + (\cos\phi)\hat{j}
\end{equation}
Thus $$\vb{L} = \frac{\hbar}{i}\left[(-\sin\phi\hat{i} + \cos\phi\hat{j})\frac{\partial}{\partial\theta}-(\cos\theta\cos\phi\hat{i}+\cos\theta\sin\phi\hat{j}-\sin\theta\hat{k})\frac{1}{\sin\theta}\frac{\partial}{\partial\phi}\right]$$ Evidently
\begin{align}
	L_x&=\frac{\hbar}{i}\left(-\sin\phi\frac{\partial}{\partial\theta}-\cos\phi\cot\theta\frac{\partial}{\partial\phi}\right)\label{4.127}\\
	L_y&=\frac{\hbar}{i}\left(+\cos\phi\frac{\partial}{\partial\theta}-\sin\phi\cot\theta\frac{\partial}{\partial\phi}\right)\label{4.128}
\end{align}
and
\begin{equation}\label{4.129}
	\boxed{L_z=\frac{\hbar}{i}\frac{\partial}{\partial\phi}}
\end{equation}
We shall also need the raising and lowering operators: $$L_{\pm}=L_x\pm iL_y=\frac{\hbar}{i}\left[(-\sin\phi\pm i\cos\phi)\frac{\partial}{\partial\theta}-(\cos\phi\pm i\sin\phi)\cot\theta\frac{\partial}{\partial\phi}\right]$$ But $\cos\phi\pm i\sin\phi=e^{\pm i\phi}$, so
\begin{equation}\label{4.130}
	L_{\pm}=\pm \hbar e^{\pm i\phi}\left(\frac{\partial}{\partial\theta}\pm i\cot\theta\frac{\partial}{\partial\phi}\right)
\end{equation}
In particular
\begin{equation}\label{4.131}
	L_+L_-=-\hbar^2\left(\frac{\partial^2}{\partial\theta^2}+\cot\theta\frac{\partial}{\partial\theta}+\cot^2\theta\frac{\partial^2}{\partial\phi^2}+i\frac{\partial}{\partial\phi}\right)
\end{equation}
and hence
\begin{equation}\label{4.132}
	\boxed{L^2=-\hbar^2\left[\frac{1}{\sin\theta}\frac{\partial}{\partial\theta}\left(\sin\theta\frac{\partial}{\partial\theta}\right)+\frac{1}{\sin^2\theta}\frac{\partial^2}{\partial\phi^2}\right]}
\end{equation}
We are now in position to determine $f_l^m(\theta,\phi)$. It's an eigenfucntion of $L^2$, with eigenvalue $\hbar^2 l(l+1)$: $$L^2f_l^m=-\hbar^2\left[\frac{1}{\sin\theta}\frac{\partial}{\partial\theta}\left(\sin\theta\frac{\partial}{\partial\theta}\right)+\frac{1}{\sin^2\theta}\frac{\partial^2}{\partial\phi^2}\right]f_l^m=\hbar^2l(l+1)f_l^m$$ But this is precisely the "angular equation" (\ref{4.18}). And it's also an eigenfucntion of $L_z$, with the eigenvalue $m\hbar$: $$L_zf_l^m=\frac{\hbar}{i}\frac{\partial}{\partial\phi}f_l^m=\hbar mf_l^m$$, but this is equivalent to the azimuthal equation (\ref{4.21}). We have already solved this system of equations: The result (appropriately normalized) is the spherical harmonic, $Y_l^m(\theta,\phi)$. \textit{Conclusion}: Spherical harmonics are eigenfunctions of $L^2$ and $L_z$. When we solved the Schrodinger equation by separation of variables, in Section \ref{sec:4.1}, we were inadvertenly constructing simultaneous eigenfunctions of the three commuting operators $H, L^2$ and $L_z$:
\begin{equation}\label{4.133}
	H\phi=E\psi,\quad L^2\psi=\hbar^2l(l+1)\psi,\quad L_z\psi=\hbar m\psi
\end{equation}
Incidentally, we can use (\ref{4.132}) to rewrite the Schrodinger equation (\ref{4.14}) more compactly: $$\frac{1}{2mr^2}\left[-\hbar^2\frac{\partial}{\partial r}\left(r^2\frac{\partial}{\partial r}\right)+L^2\right]\psi + V\psi=E\psi$$ There is a curious final twist to this story, for the \textit{algebraic} theoryof angular momentum permits $l$ (and hence also $m$) to take on \textit{half}-integer values (\ref{4.119}), whereas separation of variables yielded eigenfunctions only for \textit{integer} values (\ref{4.29}). You might suppose that the half-integer solutions are spurious, but it turns out that they are of profuound importance, as we shall see in the following sections.

\section{Spin}
In classical mechanics, a rigid object admits two kinds of angular momentum: \textbf{orbital} ($\vb{L}=\vb{r}\times\vb{p}$), associated with the motion of the center of mass, and \textbf{spin} ($\vb{S}=I\omega$), associated with motion about the center of mass. For example, the earth has orbital angular momentum attribuitable to its revolution around the sun, and spin angular momentum coming from its daily rotation about the north-south axis. In the classical econtext this distinction is largely a matter of convenience, for when you come right down to is, $\vb{S}$ is nothing but the sum total of the "orbital" angulat momenta of all the rocks and dirt clods that happens in quantum mechanics, and here the distinction is absolutely fundamental. In addition to orbital angular momentum, assocaited (in the case of hydrogen) with the motion of the electron around the nucleus (and described by the spherical harmonics), the electron also carries \textit{another} form of angular momentum, which has nothing to do with motion in space (and which is not, therefore, described by any function of the position variables $r, \theta, \phi$) but which is somewhat analogous to classical spin (and for which, therefore, we use the same word.) It doesn't pay to press this analogy too far: The electron (as far as we known) is a structureless point particle, and its spin angular momentum cannot be decomposed into orbital angular momenta of constituent parts. Suffice it to say that elementary particles carry \textbf{intrinsic} angular momentum ($\vb{S}$) in addition to their "extrinsic" angular momentum ($\vb{L}$).

The \textit{algebraic} theory of spin is a carbon copy of the theory of orbital angular momentum, beginning with the fundamental commutation realtions.
\begin{equation}\label{4.124}
	[S_x,S_y]=i\hbar S_z,\quad [S_y,S_z]=i\hbar S_x,\quad [S_z,S_x]=i\hbar S_y
\end{equation}
It follows (as before) that the eigenvectors of $S^2$ and $S_z$ satisfy\footnote{Because the eigensates of spin are not \textit{functions}, I will the "ket" notation for them. (I could have done the same in Section \ref{sec:4.3}, writing $\ket{lm}$ in place of $Y_l^m$, but in that context the function notation seems more natural.) By the way, I'm running out of letters,  so I'll use $m$ for the eigenvalue of $S_z$ just as I did for $L_z$ (some authors write $m_l$ and $m_s$ at this stage, just to be absolutely clear).}
\begin{equation}\label{4.135}
	S^2\ket{sm}=\hbar^2s(s+q)\ket{sm};\quad S_z\ket{sm}=\hbar m\ket{sm}
\end{equation}
and 
\begin{equation}\label{4.136}
	S_{\pm}\ket{sm}=\hbar\sqrt{s(s+1)-m(m\pm 1)}\ket{s(m\pm 1)}
\end{equation}
where $S_{\pm}\equiv S_xºpm iS_y$. But this time the eigenfunctions are not spherical harmonics (they're note functions of $\theta$ and $\phi$ at \textit{all}), and there is no a \textit{a priori} reason to exclude the half-integer values of $s$ and $m$:
\begin{equation}\label{4.137}
	s = 0,\frac{1}{2},1,\frac{3}{2}...;\quad m=-s, -s+1,...,s-1,s
\end{equation}
It so happens that every elementary particle has a \textit{specific and immutable} value of $s$, which we call \textbf{the spin} of that particular apecies: pi meson have spin $0$; electrons have spin $1/2$; photons have spin $1$; deltas have spin $3/2$; gravitons have spin $2$; and so on. By contrast; the \textit{orbital} angular momentum quantum number $l$ (for an electron in a hydrogen atom, say) can take on any (integer) values you please, and will changa from one to another when the system is pertubed. But $s$ is \textit{fixed}, for any given particle, and this makes the theory of spin comparatively simple.

\subsubsection{Spin 1/2}
By far the most important case is $s=1/2$, for this is the spin of the particles that make up ordinary matter (protons, neutrons, and electrons), as well as all quarks and all leptons. Moreover, once you understand spin $1/2$, it is a simple matter to work out the formalism for any higher spin. There are just \textit{two} eigenstates: $\ket{\frac{1}{2}\frac{1}{2}}$, which we call \textbf{spin up} (informally, $\uparrow$), and $\ket{\frac{1}{2}(-\frac{1}{2})}$, which we call \textbf{spin down} ($\downarrow$). Using these as basis vectors, the general state of a spin $-1/2$ particle can be expressed as a two-element column matrix (or \textbf{spinor}):
\begin{equation}\label{4.139}
	\chi=\mqty(a\\b)=a\chi_++b\chi_-
\end{equation}
with
\begin{equation}\label{4.140}
	\chi_+=\mqty(1\\0)
\end{equation}
representing spin up, and
\begin{equation}\label{4.141}
	\chi_-=\mqty(0\\1)
\end{equation}
for spin down.

Meanshile, the spin operators become $2\times 2$ matrices, which we can work out by noting their effect on $\chi_+$ and $\chi_-$. (\ref{4.135}) says
\begin{equation}\label{4.142}
	\vb{S}^2\chi_+=\frac{3}{4}\hbar^2\chi_+\quad\mbox{ and  }\quad \vb{S}^2\chi_-=\frac{3}{4}\hbar^2\chi_-
\end{equation}
If we write $\vb{S}^2$ as a matrix with (as yet) undetermined elements, $$\vb{S}^2=\mqty(c&d\\e&f)$$ then the first equation says $$\mqty(c&d\\e&f)\mqty(1\\0)=\frac{3}{4}\hbar^2\mqty(1\\0),\quad \mbox{or }\quad \mqty(c\\e)=\mqty(\frac{3}{4}\hbar^2\\0)$$ so $c=(3/4)\hbar^2$ and $e=0$. The second equation says $$\mqty(c&d\\e&f)\mqty(0\\1)=\frac{3}{4}\hbar^2\mqty(0\\1)\quad\mbox{or }\mqty(d\\f)\mqty(0\\\frac{3}{4}\hbar^2)$$ so $d=0$ and $f=(3/4)\hbar^2$. \textit{Conclusion}:
\begin{equation}\label{4.143}
	\vb{S}^2=\frac{3}{4}\hbar^2\mqty(1&0\\0&1)
\end{equation}
Similarly,
\begin{equation}\label{4.144}
	\vb{S}\chi_+=\frac{\hbar}{2}\chi_+,\quad \vb{S}_z\chi_-=-\frac{\hbar}{2}\chi_-
\end{equation}
from which it follows that
\begin{equation}\label{4.145}
	\vb{S}_z=\frac{\hbar}{2}\mqty(1&0\\0&-1)
\end{equation}
Meanwhile, (\ref{4.136}) says $$\vb{S}_+\chi_-=\hbar\chi_+\quad \vb{S}_z\chi_+\hbar\chi_-,\quad \vb{S}_+\chi_+=\vb{S}_-\chi_-=0$$ so 
\begin{equation}\label{1.146}
	\vb{S}_+=\hbar\mqty(0&1\\0&0),\quad \vb{S}_-=\hbar\mqty(0&0\\1&0)
\end{equation}
Now $S_{\pm}=S_x\pm iS_y$, so $S_x=(1/2)(S_+ + S_-)$ and $S_y=(1/2i)(S_+-S_-)$, and hence
\begin{equation}\label{4.147}
	\vb{S}_x=\frac{\hbar}{2}\mqty(0&1\\1&0),\quad \vb{S}_y=\frac{\hbar}{2}\mqty(0&-i\\i&0)
\end{equation}
Sinse $\vb{S}_x, \vb{S}_y$ and $\vb{S}_z$ all carry a factor of $\hbar/2$, it is tider to write $\vb{S}=(\hbar/2)\sigma$, where
\begin{equation}\label{4.148}\marginnote{Pauli spin matrices}
	\boxed{\sigma_x\equiv\mqty(0&1\\1&0),\quad \sigma_y\equiv\mqty(0&-i\\i&0),\quad \sigma_z\equiv\mqty(1&0\\0&-1)}
\end{equation}
These are de famous \textbf{Pauli spin matrices}. Notice thar $\vb{S}_x, \vb{S}_y, \vb{S}_z$ and $\vb{S}^2$ are all \textit{hermitian} (as they should be, since they represnet observables). On the other hand, $\vb{S}_+$ and $\vb{S}_-$ are \textit{not} hermitian--evidently they are not obervables.

Te eigenspinors of $\vb{S}_z$ are (of course):
\begin{equation}\label{4.149}
	\chi_+=\mqty(1\\0), \left(\mbox{eigenvalue $+\frac{\hbar}{2}$}\right);\quad \chi_-=\mqty(0\\1), \left(\mbox{eigenvalue $-\frac{\hbar}{2}$}\right)
\end{equation}
If you measure $S_z$ on a particle in the general state $\chi$ (\ref{4.139}), you could get $+\hbar/2$, with probability $|a|^2$, or $-\hbar/2$, with probability $|b|^2$. Since these are the \textit{only} possibilities,
\begin{equation}\label{4.150}
	|a|^2+|b|^2=1
\end{equation}
(i.e., the spinor must be \textit{normalized}).\footnote{People often say that $|a|^2$ is the "probability that the particle is in the spin-up state", but this is slppy language; what they \textit{mean} is that if you \textit{measured} $S_z$, $|a|^2$ is the probability you'd get $\hbar/2$}.

But what if, instead, you chose to measure $S_x$? What the possible results, and what are thei respective probabilities? According to the generalized statistical interpretation,  we need to know the eigenvalues and eigenspinors of $\vb{S}_x$. The characteristic equation is $$\mqty|-\lambda&\hbar/2\\\hbar/2&-\lambda|=0\Rightarrow \lambda^2=\left(\frac{\hbar}{2}\right)^2\Rightarrow \lambda =\pm\frac{\hbar}{2}$$
Not surprisongly, the possible values for $S_x$ are the same as those for $S_z$. The eigenspinor are obtained in the usual way: $$\frac{\hbar}{2}\mqty(0&1\\1&0)\mqty(\alpha\\\beta)=\pm \frac{\hbar}{2}\mqty(\alpha\\\beta)\Rightarrow \mqty(\alpha\\\beta)=\pm \mqty(\alpha\\\beta)$$, so $\beta=\pm\alpha$. Evidently the (normalized) eigenspinors of $\vb{S}_x$ are
\begin{equation}\label{4.151}
	\chi_+^{(x)}\mqty(\frac{1}{\sqrt{2}}\\\frac{1}{\sqrt{2}}),\left(\mbox{eigenvalue $+\frac{\hbar}{2}$}\right);\quad \chi_-^{(x)}=\mqty(\frac{1}{\sqrt{2}}\\\frac{-1}{\sqrt{2}}),\left(\mbox{eigenvalue $-\frac{\hbar}{2}$}\right)
\end{equation}
As the eigenvectors of hermitian matrix, they span the space; the generic spinor $\chi$ (\ref{4.139}) can be expressed as a linear combination of them:
\begin{equation}\label{4.152}
	\chi = \left(\frac{a+b}{\sqrt{2}}\right)\chi_+^{(x)} + \left(\frac{a-b}{\sqrt{2}}\right)\chi_-^{(x)}
\end{equation}
If you measure $S_x$, the probability of getting $+\hbar/2$ is $(1/2)|a+b|^2$, and the probability of getting $-\hbar/2$ is $(1/2)|a-b|^2$. (You shouls check for yourself that these probabilities add up to $1$).

I'd loke now to walk you through an imaginary mesurement scenario involving spin $1/2$, becuse it serves to ollustrate in very concrete terms some of the abstract ideas we discussed back in Chapter \ref{cap:1}. Let's say we start out with a particle in the satte $\chi_+$. If someone asks, "What is the $z$-component of that particle's spin angular momentum?", we could answer unambiguosly: $+\hbar/2$. For a measurement of $S_z$ is certain to return that value. But if our interrogator asks instead, "What is the $x$-component of that particle's spin angular momentum?" we are obliged to equivocate: If you measure $S_x$, the chances are fifty-fifty (in the sense of Section \ref{sec:1.2}), he will regard this as an inadequate--not to say impertinent--response: "Are ypu telling me that you \textit{don't know} the true state of that particle?" On the countrary; I know \textit{precisely} what the state of the particle is: $\chi_+$. "Well, then, how come you can't tell me what the $x$-component of its spin is?" Because it simply \textit{does not have} a particular $x$-component of spin. Indeed, it \textit{cannot}, for if both $S_x$ and $S_z$ were ell-defines, the uncertainty principle would be violated.

At this point our challegner grabs the test-tube and \textit{measures} the $x$-component of its spin; let's say he gets the value $+\hbar/2$. "Aha!" (he shouts it triumph), "You \textit{lied}! This particle has a perfectly well-defined value of $S_x=\hbar/2$" Well, sure--it does \textit{now} , but that doesn't prove it \textit{had} the value, prior to your measuerement. "You have obviously been reduced to splittin hairs. And anyway, what happened to your ucnertainty principle? I now know both $S_x$ \textit{and} $S_z$" I'm sorry, but you do \textit{not}: In the course of your measurement, you altered the particle's state; it is noy in the state $\chi_-^{(x)}$, and whereas you know the value of $S_x$, you no longer know the calue of $S_z$. "But I was extremely careful not to disturb the particle when I measured $S_x$". Veru well, if you don't belive me, \textit{check it out}: Measure $S_z$,  and see what you get. (Of course, he \textit{may} get $+\hbar/2$, which will be embarrasing to my case--but if we repeat this whole scenario over and over, half the time he weill get $-\hbar/2$.)

To the layman, the philosopher, or the classical physicist, a statement of the form "this particle doesn't have a well-defined position" (or momentum, or $x$-component of spin angular momentum, or whatever) sound vague, incompetent, or (worst of all) profound. It is none od these. But its precise meaning is, I think, almost impossible to convey to anyone who has not studied quantum mechanics in some depth. If you find your own comprehension slipping, from time to time (if you \textit{don't}, you probably haven't understood the problem), come back to the spin-$1/2$ system: It is the simplest and cleanest context for thinking through the conecptual paradoxes of quantum mechanics.

\subsubsection{Electron in a Magnetic Field}\label{sec:4.4.2}
A spinning charged particle constitutes a magnetic dipole. Its \textbf{magnetic dipolemoment}, $\vb*{\mu}$ is proportional to its spin angular momentum, $\vb{S}$
\begin{equation}\label{4.156}
	\vb*{\mu}=\gamma\vb{S}
\end{equation}
the proportionality constant, $\gamma$, is called the \textbf{gyromagnetic ratio}. When a magnetic dipole is placed in a magnetic field $\vb{B}$, it experiences a torque, $\vb*{\mu}\times\vb{B}$, which tends to line it up parallel to the fields (just like a compass needle). The energy assocaited with this torque is

\begin{equation}\label{4.157}
	H =-\vb*{\mu}\cdot\vb{B}
\end{equation}
so the Hamiltonian of a spinning charged particle, at rest in a magnetic field $\vb{B}$ is
\begin{equation}\label{4.158}
	H = -\gamma\vb{B}\cdot\vb{S}
\end{equation}

\begin{example}
	\textbf{Larmor precession:}
\end{example}

\begin{example}
	\textbf{The Stern-Gerlach experiment:}
\end{example}

\subsubsection{Addition of Angular Momenta}\label{sec:4.4.3}
Suppose now that we have \textit{two} spin -$1/2$ particles--for example, the electron and the proton in the ground state\footnote{I put them in the ground state so there won't be any \textit{orbital} angular momentum to worry about.} of hydrogen. Each can have spin up or spin down, so there are four possibilities in all\footnote{More precisely, each particle is in a \textit{linear combinatio} of spin up and spin down, adn the composite system is in a \textit{linear combination} of the four states listed.}
\begin{equation}\label{4.175}
	\uparrow\uparrow,\quad \uparrow\downarrow,\quad \downarrow\uparrow, \quad \downarrow\downarrow
\end{equation}

























