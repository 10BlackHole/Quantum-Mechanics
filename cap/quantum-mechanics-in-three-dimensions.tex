\chapter{Quantum Mechanics In Three Dimensions}
%\section{Schrodinger Eqaution In Spherical Coordinates}

%\section{The Hydrogen Atom}

\section{Angular Momentum}
Classically, the angular momentum of a particle (with respecto to the origin) is given by the formula
\begin{equation}\label{4.95}
	\vb{L}=\vb{r}\times\vb{p}
\end{equation}
which os to say,
\begin{equation}\label{4.96}
	L_x=yp_z-zp_y,\quad L_y=zp_x-xp_z,\quad L_z=xp_y-yp_x
\end{equation}
The corresponding quantum operators are obtained by the standard prescription $p_x\to -\hbar\partial/\partial x, p_y\to -i\hbar\partial/\partial y, p_z\to -\hbar\partial/\partial z$.

\subsubsection{Eigenvalues}
The operators $L_x$ and $L_y$ do not commute; in fact\footnote{Note that all the operators we encounter in quantum mechanics are distributive with respect to addition: $A(B+C)=AB +AC$. In particular, $[A,B+C]=[A,B]+[A,C]$.}
\begin{align}
	\nonumber [L_x,L_y]&=[yp_z-zp_y, zp_x-xp_z]\\
										 &=[yp_z,zp_x]-[yp_z,zp_z]-[zp_y,zp_x]+[zp_y,xp_z]\label{4.97}
\end{align}
From the canonical commutation relations\footnote{The cannonical commutation relations are $[r_i,p_j]=-[p_i,r_j]=i\hbar\delta_{ij}\quad [r_i,r_j]=[p_i,p_j]=0$} we know that the inly operators here that \textit{fail} to commute are $x$ with $p_x$, $y$ with $p_y$, and $z$ with $p_z$. So the two middles terms drop out, leaving
\begin{equation}\label{4.98}
	[L_x,L_y]=yp_x[p_z,z]+xp_y[z,p_z]=i\hbar(xp_y-yp_x)=i\hbar L_z
\end{equation}
Of course, we could have started out with $[L_y,L_z]$ or $[L_z,L_x]$, but there is no need to calculate these separately--we can get them immediately by cyclic permutation of the indices ($x\to y, y\to z, z\to x$)
\begin{equation}\label{4.99}\marginnote{Fundamental commutation relations for angular momentum}
	\boxed{[L_x,L_y]=i\hbar L_z;\quad [L_y,L_z]=i\hbar L_x;\quad [L_z,L_x]=i\hbar L_y}
\end{equation}
These are the fundamental commutation relations for angular momentum; everything else follows from them.

Notice that $L_x, L_y$ and $L_z$ are \textit{incompatible} obervables. According to the generalized uncertainty principle (\ref{3.62})
\begin{equation*}
	\sigma_{L_x}^2\sigma_{L_y}^2\geq\left(\frac{1}{2i}\expval{i\hbar L_z}\right)^2=\frac{\hbar^2}{4}\expval{L_z}^2
\end{equation*}
or
\begin{equation}\label{4.100}
	\sigma_{L_x}\sigma_{L_y}\geq\frac{\hbar}{2}|\expval{L_z}|
\end{equation}
It would therefore be fuitle to look for states that are simultaneoulsy eigenfunctions of $L_x$ and $L_y$. On the other hand,  the square of the total angular momentum,
\begin{equation}\label{4.101}
	L^2\equiv L_x^2+L_y^2+L_z^2
\end{equation}
does commute with $L_x$:
\begin{align*}
	[L^2,L_x]&=[L_x^2,Lx]+[L_y^2,L_x]+[L_z^2,Lx]\\
					 &=L_y[L_y,L_x]+[L_y,L_x]L_y+L_z[L_z,L_x]+[L_z,L_x]L_z\\
					 &=L_y(-i\hbar L_z)+(-i\hbar L_z)L_y+L_z(i\hbar L_y)+(i\hbar L_y)L_z\\
					 &=
\end{align*}
(I used (\ref{3.64})) to simplify the commutator; note also that any operator commutes with itself. It follows, of course, that $L^4$ also commute with $L_y$ and $L_z$:
\begin{equation}\label{4.102}
	[L^2,L_x]=0,\quad [L^2,L_y]=0,\quad {L^2,L_z}=0
\end{equation}
or, more compactly, 
\begin{equation}\label{4.103}
	[L^2,\vb{L}]=0
\end{equation}
So $L^2$ is compatible with each component of $\vb{L}$, and we can hope to find simultanoeus eigenstates of $L^2$ and (say) $L_z$:
\begin{equation}\label{4.104}
	L^2f=\lambda f\quad \mbox{and}\quad L_zf=\mu f
\end{equation}
We'll use a "ladder operator" technique, very similar to the one we applied to the harmonic oscillator back in Section \ref{sec:2.3.1}. Let
\begin{equation}\label{4.105}
	\boxed{L_{\pm}\equiv L_x\pm iL_y}
\end{equation}
The commutator with $L_z$ is
\begin{equation*}
	[L_z,L_{\pm}]=[L_z,L_x]\pm i[L_z,L_y]=i\hbar L_y\pm i(-i\hbar L_x)=\pm\hbar (L_x\pm iL_y)
\end{equation*}
so
\begin{equation}\label{4.106}
	[L_z,L{\pm}]=\pm \hbar L_{\pm}
\end{equation}
And. of course, 
\begin{equation}\label{4.107}
	[L^2,L_{\pm}]=0
\end{equation}
I claim that if $f$ es an eigenfunction of $L^2$ and $L_z$, so also is $L_{\pm} f$: (\ref{4.107}) says
\begin{equation}\label{4.108}
	L^2(L_{\pm}f)=L_{\pm}(L^2 f)=L_{\pm}(\lambda f)=\lambda(L_{\pm}f)
\end{equation}
so $L_{\pm}f$ is an eigenfunction of $L^2$, with the same eigenvalue $\lambda$, and (\ref{4.106}) says
\begin{align}
\nonumber	L_z(L_{\pm}f)&=(L_zL_{\pm}-L_{\pm}L_z)f+L_{\pm}L_zf=\pm\hbar L_{\pm}f+L_{\pm}(\mu f)\\
											 &=(\mu\pm \hbar)(L_{\pm}f)\label{4.109}
\end{align}
so $L_{\pm} f$ is an eigenfunction of $L_z$ with the \textit{new} eigenvalue $\mu\pm\hbar$. We call $L_+$ the "raising" operator, because is \textit{increases} the eigenvalue of $L_z$ by $\hbar$, and $L_-$ the "lowering" operator, because it \textit{lowers} the eigenvalue by $\hbar$.

For a given value of $\lambda$, then we obtain a "ladder" of states, with each "rung" separated from its neighbors by one unit of $\hbar$ in the eigenvalue of $L_z$. To ascend the ladder we apply the raising operator, and to descend, the lowering operator. But this procces cannot go on forever. Eventually we're going to reach a state gor which the $z$-component exceeds the \textit{total}, and that cannot be. There must exist a "top rung", such that
\begin{equation}\label{4.110}
	L_+f_t=0
\end{equation}
Let $\hbar l$ be the eigenvalue of $L_z$ at this top ring (the appropriateness of the letter "$l$" will appear in a moment):
\begin{equation}\label{4.111}
	L_zf_t=\hbar lf_t;\quad L^2f_t=\lambda f_t
\end{equation}
Now,
\begin{align*}
	L_{\pm}L_{\mp}&=(L_x\pm iL_y)(L_x\mp iL_y)=L_x^2+L_y^2\mp i(L_xL_y-L_yL_x)\\
								&=L^2-L_z^2\mp i(i\hbar L_z)
\end{align*}
or, putting in the other way around,
\begin{equation}\label{4.112}
	L^2=L_{\pm}L_{\mp}+L_z^2\mp \hbar L_z
\end{equation}
It follows that
$$L^2f_t=(L_-L_++L_z^2+\hbar L_z)f_t=(0+\hbar^2l^2+\hbar^2l)f_t=\hbar^2l(l+1)f_t$$ and hence
\begin{equation}\label{4.113}
	\lambda = \hbar^2l(l+1)
\end{equation}
This tells is the eigenvalue of $L^2$ in terms of the \textit{maximum} eigenvalue of $L_z$.

Meanwhile, there is also (for the same reason) a \textit{bottom} rung, $f_b$, such that
\begin{equation}\label{4.114}
	L_-f_b=0
\end{equation}
Let $\hbar\bar{l}$ be the eugenvalue if $L_z$ at this bottom rung:
\begin{equation}\label{4.115}
	L_zf_b=\hbar\bar{l}f_b;\quad L^2f_b=\lambda f_b
\end{equation}
Using (\ref{4.112}), we have $$L^2f_b=(L_+L_-+L_z^2-\hbar L_z)f_b=(0+\hbar^2\bar{l}^2-\hbar^2\bar{l})f_b=\hbar^2\bar{l}(\bar{l}-1)f_b$$ and therefore
\begin{equation}\label{4.116}
	\lambda = \hbar^2\bar{l}(\bar{l}-1)
\end{equation}
Comparing (\ref{4.113}) and ({4.116}), we see that $l(l+1)=\bar{l}(\bar{l}-1)$, so either $\bar{l}=l+1$ (which is absurd--the bottom rung would be higher than the top rung!) or else
\begin{equation}\label{4.117}
	\bar{l}=-l
\end{equation}
Evidently the eigenvalues of $L_z$ are $m\hbar$ (the appropaites of this letter will aso be clear in a moment) goes from $-l$ to $+l$ in a $N$ integer steps. In a particular, it follows that $l=-l+N$, and hence $l=N/2$, so $l$ must be \textit{an integer or a half-integer}. The eigenfunctions are characterized by the numbers $l$ and $m$:
\begin{equation}\label{4.118}
	\boxed{L^2f_l^m=\hbar^2l(l+1)f_l^m;\quad L_zf_l^m=\hbar mf_l^m}
\end{equation}
where 
\begin{equation}\label{4.119}
	l=0,1/2,1,3/2,...;\quad m=-l, -l+1,..., l-1, l
\end{equation}
For a given value of $l$, there are $2l+1$ different values of $m$ (i.e., $2l+1$ "rungs" on the "ladder"). 

I hope you're impressed: By \textit{purely algebraic means}, startin with the fundamental commutation relations for angular momentum (\ref{4.99}), we have determined the eigenvalues of $L^2$ and $L_z$--without ever seeing the eigenfunctions themselves! We turn now to the problem of constucting the eigenfunctions. but we're headed. I'll begin with the punch line: $f_l^m=Y_l^m$ the eigenfunctions of $L^2$ and $L_z$ are nothing but the old spherical harmonics, whics we came upon by a quite different route (that's ehy I chose the letters $l$ and $m$, of course). And I can now tell you whi the spherical harmonics are orthogonal: They are eigenfucntions of hermitian operatos ($L^2$ and $L_z$) belonging to distinct eigenvalues.

\subsubsection{Eigenfunctions}\label{sec:4.3.2}
First of all we need no rewrite $L_x, L_y$ and $L_z$ in spherical cooridinates. Now, $\vb{L}=(\hbar/i)(\vb{r}\times \nabla)$, and the gradient, in spherical cooridinates, is:
\begin{equation}\label{4.123}
	\nabla=\hat{r}\frac{\partial}{\partial r}+\hat{\theta}\frac{1}{r}\frac{\partial}{\partial \theta}+\hat{\phi}\frac{1}{r\sin\theta}\frac{\partial}{\partial \phi}
\end{equation}
meanwhile, $\vb{r}=r\hat{r}$, so $$\vb{L}=\frac{\hbar}{i}\left[r(\hat{r}\times\hat{r})\frac{\partial}{\partial r}+(\hat{r}\times\hat{\theta})\frac{\partial}{\partial\theta} + (\hat{r}\times\hat{\phi})\frac{1}{\sin\theta}\frac{\partial}{\partial\phi}\right]$$. But $(\hat{r}\times\hat{r})=0$, $(\hat{r}\times\hat{\theta})=\hat{\phi}$ , and $(\hat{r}\times\hat{\phi})=-\hat{\theta}$, and hence
\begin{equation}\label{4.124}
	\vb{L}=\frac{\hbar}{i}\left(\hat{\phi}\frac{\partial}{\partial\theta}-\hat{\theta}\frac{1}{\sin\theta}\frac{\partial}{\partial\phi}\right)
\end{equation}
The unit vectors $\hat{\theta}$ and $\hat{\phi}$ can be resolved into their cartesian components:
\begin{equation}\label{4.125}
	\hat{\theta}=(\cos\theta\cos\phi)\hat{i } + (\cos\theta\sin\phi)\hat{j} + (\sin\theta)\hat{k}
\end{equation}
\begin{equation}\label{4.126}
	\hat{\phi}=-(\sin\theta)\hat{i} + (\cos\phi)\hat{j}
\end{equation}
Thus $$\vb{L} = \frac{\hbar}{i}\left[(-\sin\phi\hat{i} + \cos\phi\hat{j})\frac{\partial}{\partial\theta}-(\cos\theta\cos\phi\hat{i}+\cos\theta\sin\phi\hat{j}-\sin\theta\hat{k})\frac{1}{\sin\theta}\frac{\partial}{\partial\phi}\right]$$ Evidently
\begin{align}
	L_x&=\frac{\hbar}{i}\left(-\sin\phi\frac{\partial}{\partial\theta}-\cos\phi\cot\theta\frac{\partial}{\partial\phi}\right)\label{4.127}\\
	L_y&=\frac{\hbar}{i}\left(+\cos\phi\frac{\partial}{\partial\theta}-\sin\phi\cot\theta\frac{\partial}{\partial\phi}\right)\label{4.128}
\end{align}
and
\begin{equation}\label{4.129}
	\boxed{L_z=\frac{\hbar}{i}\frac{\partial}{\partial\phi}}
\end{equation}
We shall also need the raising and lowering operators: $$L_{\pm}=L_x\pm iL_y=\frac{\hbar}{i}\left[(-\sin\phi\pm i\cos\phi)\frac{\partial}{\partial\theta}-(\cos\phi\pm i\sin\phi)\cot\theta\frac{\partial}{\partial\phi}\right]$$ But $\cos\phi\pm i\sin\phi=e^{\pm i\phi}$, so
\begin{equation}\label{4.130}
	L_{\pm}=\pm \hbar e^{\pm i\phi}\left(\frac{\partial}{\partial\theta}\pm i\cot\theta\frac{\partial}{\partial\phi}\right)
\end{equation}
In particular
\begin{equation}\label{4.131}
	L_+L_-=-\hbar^2\left(\frac{\partial^2}{\partial\theta^2}+\cot\theta\frac{\partial}{\partial\theta}+\cot^2\theta\frac{\partial^2}{\partial\phi^2}+i\frac{\partial}{\partial\phi}\right)
\end{equation}
and hence
\begin{equation}\label{4.132}
	\boxed{L^2=-\hbar^2\left[\frac{1}{\sin\theta}\frac{\partial}{\partial\theta}\left(\sin\theta\frac{\partial}{\partial\theta}\right)+\frac{1}{\sin^2\theta}\frac{\partial^2}{\partial\phi^2}\right]}
\end{equation}
We are now in position to determine $f_l^m(\theta,\phi)$. It's an eigenfucntion of $L^2$, with eigenvalue $\hbar^2 l(l+1)$: $$L^2f_l^m=-\hbar^2\left[\frac{1}{\sin\theta}\frac{\partial}{\partial\theta}\left(\sin\theta\frac{\partial}{\partial\theta}\right)+\frac{1}{\sin^2\theta}\frac{\partial^2}{\partial\phi^2}\right]f_l^m=\hbar^2l(l+1)f_l^m$$ But this is precisely the "angular equation" (\ref{4.18}). And it's also an eigenfucntion of $L_z$, with the eigenvalue $m\hbar$: $$L_zf_l^m=\frac{\hbar}{i}\frac{\partial}{\partial\phi}f_l^m=\hbar mf_l^m$$, but this is equivalent to the azimuthal equation (\ref{4.21}). We have already solved this system of equations: The result (appropriately normalized) is the spherical harmonic, $Y_l^m(\theta,\phi)$. \textit{Conclusion}: Spherical harmonics are eigenfunctions of $L^2$ and $L_z$. When we solved the Schrodinger equation by separation of variables, in Section \ref{sec:4.1}, we were inadvertenly constructing simultaneous eigenfunctions of the three commuting operators $H, L^2$ and $L_z$:
\begin{equation}\label{4.133}
	H\phi=E\psi,\quad L^2\psi=\hbar^2l(l+1)\psi,\quad L_z\psi=\hbar m\psi
\end{equation}
Incidentally, we can use (\ref{4.132}) to rewrite the Schrodinger equation (\ref{4.14}) more compactly: $$\frac{1}{2mr^2}\left[-\hbar^2\frac{\partial}{\partial r}\left(r^2\frac{\partial}{\partial r}\right)+L^2\right]\psi + V\psi=E\psi$$ There is a curious final twist to this story, for the \textit{algebraic} theoryof angular momentum permits $l$ (and hence also $m$) to take on \textit{half}-integer values (\ref{4.119}), whereas separation of variables yielded eigenfunctions only for \textit{integer} values (\ref{4.29}). You might suppose that the half-integer solutions are spurious, but it turns out that they are of profuound importance, as we shall see in the following sections.

\section{Spin}
In classical mechanics, a rigid object admits two kinds of angular momentum: \textbf{orbital} ($\vb{L}=\vb{r}\times\vb{p}$), associated with the motion of the center of mass, and \textbf{spin} ($\vb{S}=I\omega$), associated with motion about the center of mass. For example, the earth has orbital angular momentum attribuitable to its revolution around the sun, and spin angular momentum coming from its daily rotation about the north-south axis. In the classical econtext this distinction is largely a matter of convenience, for when you come right down to is, $\vb{S}$ is nothing but the sum total of the "orbital" angulat momenta of all the rocks and dirt clods that happens in quantum mechanics, and here the distinction is absolutely fundamental. In addition to orbital angular momentum, assocaited (in the case of hydrogen) with the motion of the electron around the nucleus (and described by the spherical harmonics), the electron also carries \textit{another} form of angular momentum, which has nothing to do with motion in space (and which is not, therefore, described by any function of the position variables $r, \theta, \phi$) but which is somewhat analogous to classical spin (and for which, therefore, we use the same word.) It doesn't pay to press this analogy too far: The electron (as far as we known) is a structureless point particle, and its spin angular momentum cannot be decomposed into orbital angular momenta of constituent parts. Suffice it to say that elementary particles carry \textbf{intrinsic} angular momentum ($\vb{S}$) in addition to their "extrinsic" angular momentum ($\vb{L}$).

The \textit{algebraic} theory of spin is a carbon copy of the theory of orbital angular momentum, beginning with the fundamental commutation realtions.
\begin{equation}\label{4.124}
	[S_x,S_y]=i\hbar S_z,\quad [S_y,S_z]=i\hbar S_x,\quad [S_z,S_x]=i\hbar S_y
\end{equation}
It follows (as before) that the eigenvectors of $S^2$ and $S_z$ satisfy\footnote{Because the eigensates of spin are not \textit{functions}, I will the "ket" notation for them. (I could have done the same in Section \ref{sec:4.3}, writing $\ket{lm}$ in place of $Y_l^m$, but in that context the function notation seems more natural.) By the way, I'm running out of letters,  so I'll use $m$ for the eigenvalue of $S_z$ just as I did for $L_z$ (some authors write $m_l$ and $m_s$ at this stage, just to be absolutely clear).}
\begin{equation}\label{4.135}
	S^2\ket{sm}=\hbar^2s(s+q)\ket{sm};\quad S_z\ket{sm}=\hbar m\ket{sm}
\end{equation}
and 
\begin{equation}\label{4.136}
	S_{\pm}\ket{sm}=\hbar\sqrt{s(s+1)-m(m\pm 1)}\ket{s(m\pm 1)}
\end{equation}
where $S_{\pm}\equiv S_xºpm iS_y$. But this time the eigenfunctions are not spherical harmonics (they're note functions of $\theta$ and $\phi$ at \textit{all}), and there is no a \textit{a priori} reason to exclude the half-integer values of $s$ and $m$:
\begin{equation}\label{4.137}
	s = 0,\frac{1}{2},1,\frac{3}{2}...;\quad m=-s, -s+1,...,s-1,s
\end{equation}
It so happens that every elementary particle has a \textit{specific and immutable} value of $s$, which we call \textbf{the spin} of that particular apecies: pi meson have spin $0$; electrons have spin $1/2$; photons have spin $1$; deltas have spin $3/2$; gravitons have spin $2$; and so on. By contrast; the \textit{orbital} angular momentum quantum number $l$ (for an electron in a hydrogen atom, say) can take on any (integer) values you please, and will changa from one to another when the system is pertubed. But $s$ is \textit{fixed}, for any given particle, and this makes the theory of spin comparatively simple.












